\documentclass{article}
\usepackage[utf8]{inputenc}
\usepackage[a4paper, total={6in, 8in}]{geometry}

\begin{document}
	\begin{center}
    
		\LARGE{\textbf{Quasi Experimental Design Project}} \\
        \vspace{1em}
        \normalsize\textbf{Jingxin Zhou} \\
        \normalsize{jz4031@nyu.edu} \\
        \normalsize\textbf{Qingyang Li} \\
        \normalsize{ql1111@nyu.edu}\\
        \vspace{1em}
        \normalsize{Advisor: Ying Lu} \\
        \vspace{1em}
        \normalsize{New York University} \\
        \normalsize{Applied Statistics For Social Science Research}
     
	\end{center}
	
    \begin{normalsize}
    
    	\section{Objective:}
        
        The objective of this project is to test the concept of "race as social construct" using an experimental design framework comparing the outcomes of AI powered face recognition algorithms versus human judgement. The problem in this project is defined as a multi-class classification problem, so the outcome of interest is a confusion matrix.
        
	   	\section{Approach:}
        
        We plan on approaching this project through several steps.  First, we will conduct research on determining factors of face recognition algorithm, which can help us to decide treatment factors for experimental design. Second, we will look for a face detection API that is capable to identify ethnicity. We will test different face detection APIs with Python programming. Third, we will apply digital image processing according to our predetermined treatment factors through Python programming. Fourth, we will recruit volunteers to get the test pictures and collect survey data through an online survey platform Qualtrics. Lastly, we will try different experimental design methods and perform analysis. Our final outcome will be a report and a presentation. 
        
    	\section{Timeline:}
        
        \begin{itemize}
        \item (February) Decide treatment factors and face detection API.
        \item (March) Finish building input-process-output (IPO) model using Python. Complete getting test pictures and survey data.
        \item (April) Finish experimental design and perform analysis based on the experiment. 
        \item (May) Finish report.
        \end{itemize}
        
    	\section{Possible Issues:}
    	\begin{itemize}
    	    \item Digital image processing is out of our field of study, we may spend more time on this topic.
    	    \item Since we are R programmer during our study, we may spend more time on learning and using Python programming to do the related work.  
    	\end{itemize}
\end{normalsize}
  
\end{document}
